\documentclass{article}
\usepackage{graphicx}
\usepackage[utf8]{inputenc}

\title{THEORIE DES LANGAGES}
\author{DESTRIBOIS JEAN et GHERTESCU MARIUS }
\date{Mai 2018}

\begin{document}

\maketitle

\section{Q..1}
\text G= (\Sigma, V, S, P) \newline
\Sigma = \{0,1,2,3,4,5,6,7,8,9,(,),+,-,\times,/ \} \newline
V = \{S, N, O, nb, opr\}\newline
P = \{ \newline
		S \rightarrow  (N)\newline
		S \rightarrow  N\newline
		N \rightarrow  nbN\newline
		N \rightarrow  nbO\newline
		N \rightarrow  nb\newline
		O \rightarrow  oprS\newline
		O)\rightarrow  )O\newline
\}\newline
	    nb \rightarrow  [0-9]\newline
		opr\rightarrow  [+,-,\times,/]\newline
		}\newline
\section{Q..3}
\subsection{DEFINIR  LE LANGAGE}
\text
	L= \{[0-9]+ ( (+,-,\times ,/)[0-9]+ )^\ast \}
\subsection{ALPHABET }
\Sigma= \{0,1,2,3,4,5,6,7,8,9,(,),+,-,\times ,/ \}

\subsection{AUTOMATE}
\includegraphics[width=\linewidth]{auromate.png}
\end{document}
